\documentclass{article}
\usepackage{exercise}
\usepackage[obeyspaces]{url}
\usepackage[dvipsnames]{xcolor}
\usepackage{graphicx}
\usepackage{listings}
\usepackage[scaled]{helvet}
\usepackage{amsmath,amssymb,amsfonts,mathtools,cancel}

% typeset in helvetica
\renewcommand*\familydefault{\sfdefault}


% QE stuff
\def\qe{{\sc Quantum ESPRESSO}}
\def\pwx{\texttt{pw.x}}
\def\cpx{\texttt{cp.x}}
\def\phx{\texttt{ph.x}}
\def\nebx{\texttt{neb.x}}
\def\configure{\texttt{configure}}
\def\PWscf{\texttt{PWscf }}
\def\PHonon{\texttt{PHonon}}
\def\CP{\texttt{CP}}
\def\PostProc{\texttt{PostProc}}
\def\NEB{\texttt{PWneb}} % to be decided
\def\make{\texttt{make}}
%%%%%%%%%%%%%%%%%%%%%%%%

\def\ipi{i-PI}
\newcommand{\hints}[1]{\emph{Hints: #1}}
\newcommand{\lstinxml}{\lstinline[language=XML]}
\newcommand{\lstinbash}{\lstinline[language=Bash]}
\usepackage[space=true]{accsupp}

\newcommand{\pdfactualhex}[3]{\newcommand{#1}{%
\BeginAccSupp{method=hex,ActualText=#2}#3\EndAccSupp{}}}

% PASTABLE lstlisting code
\pdfactualhex{\pdfactualdspace}{2020}{\textperiodcentered\textperiodcentered}
\pdfactualhex{\pdfactualsquote}{27}{'}
\pdfactualhex{\pdfactualbtick}{60}{`}

\lstset{tabsize=4,basicstyle=\ttfamily,breaklines=true,columns=flexible,emptylines=10000}
\lstset{literate={'}{\pdfactualsquote}1
                 {`}{\pdfactualbtick}1
                 {\ \ }{\pdfactualdspace}2
}

\lstset{
    basicstyle=\ttfamily,
    keywordstyle=\color{BrickRed},
    commentstyle=\color{Gray},
    stringstyle=\color{black},
    emphstyle=\color{RedOrange},
    columns=flexible,
    showstringspaces=false,
    xleftmargin=1em,
    deletekeywords={bin,all}
}


\lstdefinelanguage{Bash}
{
   alsodigit={-,.},
   morekeywords={ cp2k.popt, lmp_ubuntu, i-pi, vmd, for, done, python, mpirun, touch, tail, trajworks, autocorr, awk, i-pi-mergebeadspdb, grep, head, cp, cd, cat, ls, pwd, bash, sed, mkdir, vi, sh}
   %morecomment=[s][\color{orange}]{#}{\}
}

%\lstdefinelanguage{Python}
%{
%    morecomment=[s][\color{orange}]{#}{\}
%}


\lstdefinelanguage{XML}
{
  morestring=[b][\color{RedOrange}]",
%  morestring=[s]{>}{<},
  morecomment=[s]{<?}{?>},
  morecomment=[s][\color{orange}]{<!--}{-->},
  identifierstyle=\color{BlueViolet},
  keywordstyle=\color{ForestGreen},
  stringstyle=\color{RedOrange},
%  tagstyle=\color{Blue},
  morekeywords={xmlns,version,type, mode, units, forcefield, filename, stride, overwrite, nbeads,prefix}% list your attributes here
}

\lstdefinestyle{XML}
{
\tagstyle=\color{Blue}
}

\title{Computing instanton rates using \ipi{}}
\author{Yair Litman, Wei Fang and Jeremy O.\ Richardson}
\date{June 2018}

\begin{document}

\maketitle

% INTRO
In this exercises we will use \ipi{} not to run molecular dynamics
but instead to optimize stationary points on the potential-energy surface.
Then by simple extension, the ring-polymer instanton will be found
which can be used to compute the thermal rate of a chemical reaction including tunnelling effects.

The example which we will use is for the gas-phase bimolecular scattering reaction of H + CH$_4$.
This requires multiple runs of \ipi{} for the different stationary points on the surface
followed by one postprocessing calculation to combine all the data to compute the rate.

Full details of the instanton approach are described in \cite{InstReview}.

A typical work flow is summarized as follows:
\begin{enumerate}
	\item Optimize the minimum or minima and perform normal-mode analysis.
	\item Optimize the transition state and perform normal-mode analysis.
	\item Make an initial guess for an $N=16$ configuration.
	\item Compute Hessians of each bead. (This can be done approximately.)
	\item Optimize $N=16$ instanton using eigenvector following, computing gradients at each iteration, and using the Powell update formula on the ring-polymer Hessian.
	\item Double number of beads by interpolation of positions and Hessians to give new guess configuration.
	\item Optimize with Newton-Raphson, computing gradients at each iteration, and using the Powell update formula on the ring-polymer Hessian.
		\label{opt}
	\item Recompute Hessians for each bead.  (This must be done accurately.)
	\item Evaluate the fluctuation terms and hence the rate, $k$.
	\item Return to step \ref{opt} until the rate converges.
\end{enumerate}

If rates are required at more than one temperature,
it is recommended to start with those closest to the crossover temperature and cool sequentially,
again using initial guesses from the previous optimizations.

\begin{Exercise}[label={reactant},title={Optimizing and analysing the reactant}]

\Question
Go to the folder \url{input/reactant/minimization}.
A good initial guess has been provided in \url{init.xyz} and the input file \url{input.xml}.
Look at the input file which should be fairly self explanatory.  Note that the text between \url{<!--} and \url{-->} is a comment.

Then run \ipi{} using
\begin{lstlisting}[language=bash]
$ i-pi input.xml &
\end{lstlisting}
and the driver using
\begin{lstlisting}[language=bash]
$ i-pi-driver -m ch4hcbe -u
\end{lstlisting}

The simulation takes 31 steps and the final geometry can be seen in the last frame in \url{min.xc.xyz}.
Once you finish this exercise, you could come back here and try to get the optimization to finish in fewer steps, e.g. by changing \url{sd} to one of the other optimizers.

Note that as this is a bimolecular reaction, the H and the CH4 should be well separated and we have used a very large box size.
We have used the CBE potential-energy surface \cite{Corchado2009HCH4}.

\Question
Next, go to the folder \url{input/reactant/phonons} and copy the optimized geometry just obtained (last 8 lines) into a new file called \url{init.xyz}.
We will now compute the hessian of this minimum.
Run
\begin{lstlisting}[language=bash]
$ i-pi input.xml &
$ i-pi-driver -m ch4hcbe -u
\end{lstlisting}

The hessian is saved in \url{phonons.hess} and its eigenvalues in \url{phonons.eigval}.  Check that it has the required number (9) of almost zero eigenvalues.  Why?

\end{Exercise}

\begin{Exercise}[label={TS},title={Optimizing and analysing the transition state}]

Go to \url{input/TS} and optimize the transition state using the usual run commands.
Here \ipi{} treats the TS search as a one-bead instanton optimization.
The simulation takes 12 steps and puts the optimized geometry in \url{INSTANTON_FINAL_12.xyz}
and its Hessian in \url{INSTANTON_FINAL.hess_12}.
Visualize the geometry using vmd.

\end{Exercise}

\begin{Exercise}[label={inst1},title={First instanton optimization}]

\Question
Go to the folder \url{input/instanton/40}.
Copy the optimized transition state geometry obtained in Ex.~\ref{TS} and name it \url{init.xyz}.
Also copy the transition state hessian and name it \url{hessian.dat}.

Run \ipi{} in the usual way, but here we can run 4 instances of driver simultaneously using:
\begin{lstlisting}[language=bash]
$ i-pi-driver -m ch4hcbe -u &
$ i-pi-driver -m ch4hcbe -u &
$ i-pi-driver -m ch4hcbe -u &
$ i-pi-driver -m ch4hcbe -u &
\end{lstlisting}

The program first generates the initial instanton guess
based on points spread around the TS in the direction of the imaginary mode.
It then an optimization which takes 8 steps.

The instanton geometry is saved in \url{INSTANTON_FINAL_7.xyz} and should be visualized with vmd.
Finally hessians for each bead are also computed and saved in \url{INSTANTON_FINAL.hess_7}
in the shape (3N,3Nn).

\end{Exercise}

\begin{Exercise}[label={inst2},title={Second and subsequent instanton optimizations}]

\Question
Go to the folder \url{input/instanton/80}.
\Question
Copy the optimized transition state geometry obtained in Ex.~\ref{TS} and name it \url{init.xyz}.
\Question
Copy the optimized instanton geometry obtained in Ex.~\ref{inst1} and name it \url{init0}.
\Question
Copy the last hessian obtained in D and name it \url{hess0}.
\Question
Interpolate the instanton and the hessian to 80 beads by typing:
\begin{lstlisting}[language=bash]
$ python ${ipi-path}/tools/py/Instanton_interpolation.py -m -xyz init0 -hess hess0 -n 80
\end{lstlisting}
\Question
Rename the new hessian and instanton geometry to hessian.dat and init.xyz respectively
\Question
Copy the 'input.xml' file from input/instanton/40
\Question
Change the number of beads from 40 to 80 (input.xml)
\Question
Change the hessian shape from (18,18) to (18,1440) (input.xml)

Run as before.  The program performs a optimization which takes 2 steps and then computes a hessian.

\end{Exercise}

\begin{Exercise}[label={post},title={Postprocessing for the rate calculation}]

So far, we have been working with half ring polymers, but for the rate calculation, one should consider full ring polymers.
Therefore, the path with 80 beads, actually represents a ring polymer of 160 beads.
This is a little confusing currently, and may be changed in the future, so check the help (using \url{-h}) to find out what is needed.

\Question
To compute the CH4 partition function, go the \url{input/reactant/phonons} folder and type
\begin{lstlisting}[language=bash]
$ python ${ipi-path}/tools/py/Instanton_postproc.py RESTART -c reactant -t 300 -n 80 -f 5
\end{lstlisting}
which computes the ring polymer parition function for CH4 with $N=80$.
Look at the output and make a note of the translational, rotational and vibrational partition functions

\Question
To compute the TS partition function, go to \url{input/TS} and type
\begin{lstlisting}[language=bash]
$ python ${ipi-path}/tools/py/Instanton_postproc.py RESTART -c TS -t 300 -n 80
\end{lstlisting}
which computes the ring polymer parition function for the TS with $N=80$.
%(check data.out file in the corresponding inside 'solution')

\Question
To compute the instanton partition function, $B_N$ and action,
go to \url{input/instanton/80} and type
\begin{lstlisting}[language=bash]
$ python ${ipi-path}/tools/py/Instanton_postproc.py RESTART -c instanton -t 300
\end{lstlisting}
%(check data.out file in the corresponding inside 'solution')

Then it is a simple matter to combine the partition functions, BN, S etc into the formula given for the rate.
Compare the instanton results with those of the transition state in order to compute the tunnelling factor,
\begin{align}
	\kappa = \sqrt\frac{2\pi NB_N}{\beta \hbar^2}
		f_\text{trans} f_\text{rot} f_\text{vib}
		\mathrm{e}^{-S/\hbar+\beta V^\ddag}
\end{align}
where
\begin{align}
	f_\text{trans} &= \frac{Q_\text{trans}^\text{inst}}{Q_\text{trans}^\text{TS}} \\
	f_\text{rot} &= \frac{Q_\text{rot}^\text{inst}}{Q_\text{rot}^\text{TS}} \\
	f_\text{vib} &= \frac{Q_\text{vib}^\text{inst}}{Q_\text{vib}^\text{TS}}
\end{align}
In this way, you should find that the rate is about 9.8 times faster due to tunnelling.

\end{Exercise}


\bibliographystyle{unsrt}
\bibliography{biblio}
\end{document}
